\chapter{BKNR Datastore}

\vbox{
    \centering
    \includegraphics{datastoreicon}
    \vspace{1cm}}

\section{ Introduction}



\subsection{ The prevalence model}

The BKNR datastore is a persistence solution for Lisp data. It
uses the prevalence model, which is based on the following
assumptions:

All data is held in RAM.

Data can be saved to disk at once into a snapshot file and is read
from that file at startup time.

Changes to persistent data are written to a transaction log file
immediately, which can be replayed to restore all changes that
occured since the last snapshot was saved.

Every kind of operation that needs to be logged is called a
``transaction'', and such transactions are made explicit in the
program code. This is different from object-oriented databases,
where the fundamental transactions are object creation, object
deletion and slot access, which are not special cases in the
prevalence model at all.

Isolation of transactions is achieved using thread locks. In the
simplest model used by the `mp-store', transactions are serialized
using a global lock.

The transaction system is responsible for providing replay of
committed transactions after a server crash, but not for rollback
of failed transactions in a running server, except that failing
transactions are simply not logged onto disk. To roll back
transactions at points where exceptions might be excepted, use
ordinary Lisp programming techniques involving `unwind-protect'
and similar.



\subsection{ BKNR Datastore Design}

The design of the datastore aims to make explicit the
orthogonality of object system access (unlogged) and logging of
transactions (essentially independent of the object system). The
interface between transaction system and object system is
documented and allows for the implementation of alternative object
systems. For example, the blob subsystem is using the same
interface as the object subsystem.

Previous versions of the BKNR Datastore allowed the creation of
multiple datastores in a single LISP process. However, this
feature was seldom used, and could be very confusing while
developing applications. The new version of the BKNR Datastore
supports only a single datastore, which is referenced by the
special variable `*STORE*'.



\subsection{ BKNR Object Datastore}

In addition to the transaction layer (in the file `txn.lisp'), the
BKNR datastore provides persistent CLOS objects using the
Metaobject Protocol. It provides a metaclass with which slots can
be defined as persistent (stored on snapshot) or transient. The
metaclass also prohibits slot accesses outside transactions,
provides unique IDs for all objects, and provides standard query
functions like `STORE-OBJECTS-WITH-CLASS' and
`STORE-OBJECTS-OF-CLASS'. The object datastore can be seamlessly
combined with BKNR indices and XML import/export.


\section{ Obtaining and loading BKNR Datastore}

You can obtain the current CVS sources of BKNR by following the
instructions at `http://bknr.net/blog/bknr-devel'. Add the `experimental'
directory of BKNR to your `asdf:*central-registry*', and load the
indices module by evaluating the following form:

\begin{Verbatim}[fontsize=\small,frame=leftline,framerule=0.9mm,rulecolor=\color{gray},framesep=5.1mm,xleftmargin=5mm,fontfamily=cmtt]
(asdf:oos 'asdf:load-op :bknr-datastore)
\end{Verbatim}
Then switch to the `bknr.datastore' package to try out the tutorial.

\begin{Verbatim}[fontsize=\small,frame=leftline,framerule=0.9mm,rulecolor=\color{gray},framesep=5.1mm,xleftmargin=5mm,fontfamily=cmtt]
(in-package :bknr.datastore)
\end{Verbatim}


\section{ A transaction system example}
The first datastore we will build is very simple. We have a
counter variable for the store, and this counter variable can be
decremented and indecremented. We want this variable to be
persistent, so decrementing and incrementing it has to be done
through transactions that will be logged by the datastore. We also
define a `:BEFORE' method for the generic function `RESTORE-STORE'
to set the counter to `0' initially. This method will be called
every time the store is created or restored from disk.

\begin{Verbatim}[fontsize=\small,frame=leftline,framerule=0.9mm,rulecolor=\color{gray},framesep=5.1mm,xleftmargin=5mm,fontfamily=cmtt]
(defclass tutorial-store (mp-store)
  ((counter :initform 0 :accessor tutorial-store-counter)))

(defmethod restore-store :before ((store tutorial-store) &key until)
  (declare (ignore until))
  (setf (tutorial-store-counter store) 0))
\end{Verbatim}
The two transactions are declared like normal functions, but using
the `DEFTRANSACTION' macro. 

\begin{Verbatim}[fontsize=\small,frame=leftline,framerule=0.9mm,rulecolor=\color{gray},framesep=5.1mm,xleftmargin=5mm,fontfamily=cmtt]
(deftransaction incf-counter ()
  (incf (tutorial-store-counter *store*)))

(deftransaction decf-counter ()
  (decf (tutorial-store-counter *store*)))
\end{Verbatim}
When looking at the macro-expanded form of `DEFTRANSACTION', we
see that `DEFTRANSACTION' defines two functions, a toplevel
function that creates a transaction object and calls the method
`EXECUTE' on it, and a function that contains the actual
transaction code and that will be called in the context of the
transaction, and logged to disk.

\begin{Verbatim}[fontsize=\small,frame=leftline,framerule=0.9mm,rulecolor=\color{gray},framesep=5.1mm,xleftmargin=5mm,fontfamily=cmtt]
(PROGN
 (DEFUN TX-DECF-COUNTER ()
   (UNLESS (IN-TRANSACTION-P) (ERROR 'NOT-IN-TRANSACTION))
   (DECF (TUTORIAL-STORE-COUNTER *STORE*)))
 (DEFUN DECF-COUNTER (&REST #:G3047)
   (EXECUTE
    (MAKE-INSTANCE 'TRANSACTION
                   :FUNCTION-SYMBOL
                   'TX-DECF-COUNTER
                   :TIMESTAMP
                   (GET-UNIVERSAL-TIME)
                   :ARGS
                   #:G3047))))
\end{Verbatim}
The new datastore only supports a single datastore instance per
LISP session. When creating a `STORE' object, the `*STORE*'
special variable is modified to point to the datastore. Thus, we
can create our simple datastore by creating an object of type
`TUTORIAL-STORE'. The transaction log will be store in the
directory ``/tmp/tutorial-store''.

\begin{Verbatim}[fontsize=\small,frame=leftline,framerule=0.9mm,rulecolor=\color{gray},framesep=5.1mm,xleftmargin=5mm,fontfamily=cmtt]
(make-instance 'tutorial-store :directory "/tmp/tutorial-store/"
          :subsystems nil)
; Warning:  restoring #<TUTORIAL-STORE DIR: "/tmp/tutorial-store/">
; => #<TUTORIAL-STORE DIR: "/tmp/tutorial-store/">

(tutorial-store-counter *store*)
; => 0
(incf-counter)
; => 1
(incf-counter)
; => 2
(decf-counter)
; => 1
\end{Verbatim}
The three transactions have been logged to the transaction log in
``/tmp/tutorial-store/'', as we can see:

\begin{Verbatim}[fontsize=\small,frame=leftline,framerule=0.9mm,rulecolor=\color{gray},framesep=5.1mm,xleftmargin=5mm,fontfamily=cmtt]
(with-open-file (s "/tmp/tutorial-store/current/transaction-log"
             :direction :input)
        (file-length s))
; => 126
(incf-counter)
; => 2
(with-open-file (s "/tmp/tutorial-store/current/transaction-log"
             :direction :input)
        (file-length s))
; => 168
\end{Verbatim}
The transaction log is kept in a directory called ``current'', which
is where the currently active version of the snapshots and log
files are kept. When a datastore is snapshotted, the ``current''
directory is backupped to another directory with the current date,
and snapshots are created in the new ``current'' directory. However,
we cannot snapshot our tutorial datastore, as we cannot snapshot
the persistent data (the counter value). 

\begin{Verbatim}[fontsize=\small,frame=leftline,framerule=0.9mm,rulecolor=\color{gray},framesep=5.1mm,xleftmargin=5mm,fontfamily=cmtt]
(snapshot)
; => Error in function (METHOD SNAPSHOT-STORE NIL (STORE)):
; => Cannot snapshot store without subsystems...
; => [Condition of type SIMPLE-ERROR]
\end{Verbatim}
We can close the store by using the function `CLOSE-STORE'.

\begin{Verbatim}[fontsize=\small,frame=leftline,framerule=0.9mm,rulecolor=\color{gray},framesep=5.1mm,xleftmargin=5mm,fontfamily=cmtt]
*store*
; => #<TUTORIAL-STORE DIR: "/tmp/tutorial-store/">
(close-store)
; => NIL
*store*
; => NIL
\end{Verbatim}
The store can then be recreated, and the transaction log will be
read and executed upon restore.

\begin{Verbatim}[fontsize=\small,frame=leftline,framerule=0.9mm,rulecolor=\color{gray},framesep=5.1mm,xleftmargin=5mm,fontfamily=cmtt]
(make-instance 'tutorial-store :directory "/tmp/tutorial-store/"
          :subsystems nil)

; Warning:  restoring #<TUTORIAL-STORE DIR: "/tmp/tutorial-store/">
; Warning:  loading transaction log
; /tmp/tutorial-store/current/transaction-log
; => #<TUTORIAL-STORE DIR: "/tmp/tutorial-store/">
(tutorial-store-counter *store*)
; => 2
\end{Verbatim}
The store can also be restored in a later LISP session. Make sure
that all the code necessary to the execution of the transaction
log has been loaded before restoring the datastore. A later
version of the datastore will log all the code necessary in the
datastore itself, so that code and data are synchronized.


\subsection{ Debugging the datastore}

By setting the `*STORE-DEBUG*' special variable to `T', the
datastore prints a lot of useful warnings. For example
You can also restore to a certain point in time, by specifying the
`UNTIL' argument of `RESTORE-STORE'.

\begin{Verbatim}[fontsize=\small,frame=leftline,framerule=0.9mm,rulecolor=\color{gray},framesep=5.1mm,xleftmargin=5mm,fontfamily=cmtt]
(setf *store-debug* t)
; => T
(restore-store *store*)
; Warning:  restoring #<TUTORIAL-STORE DIR: "/tmp/tutorial-store/">
; Warning:  loading transaction log
; /tmp/tutorial-store/current/transaction-log
; executing transaction #$(TX-INCF-COUNTER) at timestamp 3309258381
; executing transaction #$(TX-INCF-COUNTER) at timestamp 3309258383
; executing transaction #$(TX-DECF-COUNTER) at timestamp 3309258387
; executing transaction #$(TX-INCF-COUNTER) at timestamp 3309258390
; => :NORMAL
(tutorial-store-counter *store*)
; => 2
(restore-store *store* :until 3309258387)
; Warning:  restoring #<TUTORIAL-STORE DIR: "/tmp/tutorial-store/">
; Warning:  loading transaction log
; /tmp/tutorial-store/current/transaction-log
; executing transaction #$(TX-INCF-COUNTER) at timestamp 3309258381
; executing transaction #$(TX-INCF-COUNTER) at timestamp 3309258383
; executing transaction #$(TX-DECF-COUNTER) at timestamp 3309258387
; => :NORMAL
(tutorial-store-counter *store*)
; => 1
\end{Verbatim}


\subsection{ Adding a subsystem}
Now that we can restore the counter state by loading the
transaction log, we want to add a subsystem to be able to snapshot
the state of the counter. Thus, we won't need to execute every
single incrementing or decrementing transaction to restore our
persistent state.
To do this, we have to create a store-subsystem that will be able
to write the counter number to a file and to reload it on restore.

\begin{Verbatim}[fontsize=\small,frame=leftline,framerule=0.9mm,rulecolor=\color{gray},framesep=5.1mm,xleftmargin=5mm,fontfamily=cmtt]
(defclass counter-subsystem ()
  ())
\end{Verbatim}
Three methods are used to interact with the subsystem.
The first method is `INITIALIZE-SUBSYSTEM', which is called after
the store has been created and restored. It is used to initialize
certain parameters of the subsystem. We won't use this method
here, as our subsystem is very simple.
The second method is `SNAPSHOT-SUBSYSTEM', which is called when
the store is snapshotted. The subsystem has to store the
persistent data it handles to a snapshot file inside the current
directory of the store. Our `COUNTER-SUBSYSTEM' writes the current
value of the counter to a file named ``counter'' in the current
directory of the store (the old directory has been renamed).

\begin{Verbatim}[fontsize=\small,frame=leftline,framerule=0.9mm,rulecolor=\color{gray},framesep=5.1mm,xleftmargin=5mm,fontfamily=cmtt]
(defmethod snapshot-subsystem ((store tutorial-store)
                (subsystem counter-subsystem))
  (let* ((store-dir (ensure-store-current-directory store))
    (counter-pathname
     (make-pathname :name "counter" :defaults store-dir)))
    (with-open-file (s counter-pathname :direction :output)
      (write (tutorial-store-counter store) :stream s))))
\end{Verbatim}
Finally, the method `RESTORE-SUBSYSTEM' is called at restore time
to tell the subsystem to read back its persistent state from the
current directory of the store. Our `COUNTER-SUBSYSTEM' reads back
the counter value from the file named ``counter''. If it can't find
the file (for example if this is the first time that our datastore
is created, the file won't be there, so we issue a warning and set
the counter value to 0.

\begin{Verbatim}[fontsize=\small,frame=leftline,framerule=0.9mm,rulecolor=\color{gray},framesep=5.1mm,xleftmargin=5mm,fontfamily=cmtt]
(defmethod restore-subsystem ((store tutorial-store)
               (subsystem counter-subsystem) &key
               until)
  (declare (ignore until))
  (let* ((store-dir (ensure-store-current-directory store))
    (counter-pathname
     (make-pathname :name "counter" :defaults store-dir)))
    (if (probe-file counter-pathname)
   (with-open-file (s counter-pathname :direction :input)
     (let ((counter (read s)))
       (setf (tutorial-store-counter store) counter)))
   (progn
     (warn "Could not find store counter value, setting to 0.")
     (setf (tutorial-store-counter store) 0)))))
\end{Verbatim}
Now we can close our current store, and instantiate it anew with a
`COUNTER-SUBSYSTEM'.

\begin{Verbatim}[fontsize=\small,frame=leftline,framerule=0.9mm,rulecolor=\color{gray},framesep=5.1mm,xleftmargin=5mm,fontfamily=cmtt]
(close-store)
; => NIL
(make-instance 'tutorial-store :directory "/tmp/tutorial-store/"
          :subsystems (list (make-instance 'counter-subsystem)))
; Warning:  restoring #<TUTORIAL-STORE DIR: "/tmp/tutorial-store/">
; Warning:  Could not find store counter value, setting to 0.
; Warning:  loading transaction log
; /tmp/tutorial-store/current/transaction-log
; => #<TUTORIAL-STORE DIR: "/tmp/tutorial-store/">
(snapshot)
; => NIL
(restore)
; Warning:  restoring #<TUTORIAL-STORE DIR: "/tmp/tutorial-store/">
; => :NORMAL
\end{Verbatim}


\section{ An object store example}
The BKNR object datastore is implemented using a special subsystem
`STORE-OBJECT-SUBSYSTEM'. Every object referenced by the store
object subsystem has a unique ID, and must be of the class
`STORE-OBJECT'. The ID counter in the store-object subsystem is
incremented on every object creation.

All store objects have to be of the metaclass `PERSISTENT-CLASS',
which will ensure the object is referenced in the base indices of
the object datastore, and that slot access is only done inside a
transaction. The subsystem makes heavy use of BKNR indices, and
indexes object by ID and by class.
The ID index can be queried using the functions
`STORE-OBJECT-WITH-ID', which returns the object with the
requested ID, `ALL-STORE-OBJECTS' which returns all current store
objects, and `MAP-STORE-OBJECTS', which applies a function
iteratively to each store object. The class index can be queried
using the functions `ALL-STORE-CLASSES', which returns the names
of all the classes currently present in the datastore, and
`STORE-OBJECTS-WITH-CLASS', which returns all the objects of a
specific class (across superclasses also, so
`(STORE-OBJECTS-WITH-CLASS $\backslash$'STORE-OBJECT)' returns all the
existing store objects.


\subsection{ Store and object creation}
We can create an object datastore by creating a `STORE' with the
subsystem `STORE-OBJECT-SUBSYSTEM'.

\begin{Verbatim}[fontsize=\small,frame=leftline,framerule=0.9mm,rulecolor=\color{gray},framesep=5.1mm,xleftmargin=5mm,fontfamily=cmtt]
(make-instance 'mp-store :directory "/tmp/object-store/"
          :subsystems (list
             (make-instance 'store-object-subsystem)))

; Warning:  restoring #<MP-STORE DIR: "/tmp/object-store/">
; => #<MP-STORE DIR: "/tmp/object-store/">
(all-store-objects)
; => NIL
\end{Verbatim}
We can now create a few store objects (which is not very
interesting in itself). Store objects have to be created inside a
transaction so that the object creation is logged into the
transaction log. This is done by using the transaction
`MAKE-OBJECT'. The transaction also automatically gets a unique ID
from the store object subsystem.

\begin{Verbatim}[fontsize=\small,frame=leftline,framerule=0.9mm,rulecolor=\color{gray},framesep=5.1mm,xleftmargin=5mm,fontfamily=cmtt]
(make-object 'store-object)
; => #<STORE-OBJECT ID: 0>
(make-object 'store-object)
; => #<STORE-OBJECT ID: 1>
(all-store-objects)
; => (#<STORE-OBJECT ID: 0> #<STORE-OBJECT ID: 1>)
(all-store-classes)
; => (STORE-OBJECT)
\end{Verbatim}
Object deletion also has to be done through the transaction
`DELETE-OBJECT', which will log the deletion of the object in the
transaction log, and remove the object from all its indices.

\begin{Verbatim}[fontsize=\small,frame=leftline,framerule=0.9mm,rulecolor=\color{gray},framesep=5.1mm,xleftmargin=5mm,fontfamily=cmtt]
(make-object 'store-object)
; executing transaction #$(TX-MAKE-OBJECT STORE-OBJECT)
; at timestamp 3309260107
; => #<STORE-OBJECT ID: 12>
(store-object-with-id 12)
; => #<STORE-OBJECT ID: 12>
(delete-object (store-object-with-id 12))
; executing transaction #$(TX-DELETE-OBJECT 12)
; at timestamp 3309260112
; => T
(store-object-with-id 12)
; => NIL
\end{Verbatim}


\subsection{ Defining persistent classes}
A more interesting thing is to create our own persistent class,
which we will call `TUTORIAL-OBJECT'.

\begin{Verbatim}[fontsize=\small,frame=leftline,framerule=0.9mm,rulecolor=\color{gray},framesep=5.1mm,xleftmargin=5mm,fontfamily=cmtt]
(defclass tutorial-object (store-object)
  ((a :initarg :a :reader tutorial-object-a))
  (:metaclass persistent-class))
\end{Verbatim}
We can also use the `DEFINE-PERSISTENT-CLASS' to define the class
`TUTORIAL-OBJECT':

\begin{Verbatim}[fontsize=\small,frame=leftline,framerule=0.9mm,rulecolor=\color{gray},framesep=5.1mm,xleftmargin=5mm,fontfamily=cmtt]
(define-persistent-class tutorial-object ()
  ((a :read)))
\end{Verbatim}
This gets macroexpanded to the following form. The `EVAL-WHEN' is
there to ensure timely definition of the accessor methods.

\begin{Verbatim}[fontsize=\small,frame=leftline,framerule=0.9mm,rulecolor=\color{gray},framesep=5.1mm,xleftmargin=5mm,fontfamily=cmtt]
(EVAL-WHEN (:COMPILE-TOPLEVEL :LOAD-TOPLEVEL :EXECUTE)
  (DEFCLASS TUTORIAL-OBJECT
            (STORE-OBJECT)
            ((A :READER TUTORIAL-OBJECT-A :INITARG :A))
            (:METACLASS PERSISTENT-CLASS)))
\end{Verbatim}
We can now create a few instance of `TUTORIAL-OBJECT':

\begin{Verbatim}[fontsize=\small,frame=leftline,framerule=0.9mm,rulecolor=\color{gray},framesep=5.1mm,xleftmargin=5mm,fontfamily=cmtt]
(make-object 'tutorial-object :a 2)
; => #<TUTORIAL-OBJECT ID: 3>
(make-object 'tutorial-object :a 2)
; => #<TUTORIAL-OBJECT ID: 4>
(make-object 'tutorial-object :a 2)
; => #<TUTORIAL-OBJECT ID: 5>

(store-object-with-id 5)
; => #<TUTORIAL-OBJECT ID: 5>

(all-store-classes)
; => (STORE-OBJECT TUTORIAL-OBJECT)

(store-objects-with-class 'tutorial-object)
; => (#<TUTORIAL-OBJECT ID: 3> #<TUTORIAL-OBJECT ID: 4>
;     #<TUTORIAL-OBJECT ID: 5>)
(store-objects-with-class 'store-object)
; => (#<STORE-OBJECT ID: 0> #<STORE-OBJECT ID: 1>
;  #<FOO ID: 2> #<TUTORIAL-OBJECT ID: 3>
;  #<TUTORIAL-OBJECT ID: 4> #<TUTORIAL-OBJECT ID: 5>)
\end{Verbatim}
A basic transaction used to work on persistent objects is the
transaction `CHANGE-SLOT-VALUES', which sets the values of slots
in an object. The value of a persistent slot can not be changed
outside of a transaction, as restoring the datastore would not
change the slot value.

\begin{Verbatim}[fontsize=\small,frame=leftline,framerule=0.9mm,rulecolor=\color{gray},framesep=5.1mm,xleftmargin=5mm,fontfamily=cmtt]
(define-persistent-class tutorial-object2 ()
  ((b :update)))

(make-object 'tutorial-object2 :b 3)
; executing transaction #$(TX-MAKE-OBJECT TUTORIAL-OBJECT2 B 3)
; at timestamp 3309263046
; => #<TUTORIAL-OBJECT2 ID: 16>
(setf (slot-value (store-object-with-id 16) 'b) 4)
; => Error
(change-slot-values (store-object-with-id 16) 'b 4)
; executing transaction #$(TX-CHANGE-SLOT-VALUES
; #<TUTORIAL-OBJECT2 ID: 16> B 4) at timestamp 3309263109
; => NIL
(tutorial-object2-b (store-object-with-id 16))
; => 4
\end{Verbatim}


\subsection{ Object creation and deletion protocol}
Persistent objects have the metaclass `PERSISTENT-CLASS', and have
to be created using the function `MAKE-OBJECT'. This creates an
instance of the object inside a transaction, sets its ID slot
appropriately, and then calls `INITIALIZE-PERSISTENT-INSTANCE' and
`INITIALIZE-TRANSIENT-INSTANCE'. The first method is called when
the object is created inside a transaction, but not if the object
is being restored from the snapshot file. This method has to be
overridden in order to initialize persistent
slots. `INITIALIZE-TRANSIENT-INSTANCE' is called at object
creation inside a transaction and at object creation during
restore. It is used to initialize the transient slots (not logged
to the snapshot file) of a persistent object.

We can define the following class with a transient and a
persistent slot.

\begin{Verbatim}[fontsize=\small,frame=leftline,framerule=0.9mm,rulecolor=\color{gray},framesep=5.1mm,xleftmargin=5mm,fontfamily=cmtt]
(define-persistent-class protocol-object ()
  ((a :update :transient t)
   (b :update)))
\end{Verbatim}
We can modify the slot `A' outside a transaction:

\begin{Verbatim}[fontsize=\small,frame=leftline,framerule=0.9mm,rulecolor=\color{gray},framesep=5.1mm,xleftmargin=5mm,fontfamily=cmtt]
(make-object 'protocol-object :a 1 :b 2)
; executing transaction #$(TX-MAKE-OBJECT PROTOCOL-OBJECT A 1 B 2)
; at timestamp 3309262613
; => #<PROTOCOL-OBJECT ID: 14>
(setf (protocol-object-a *) 2)
; => 2
\end{Verbatim}
However, we cannot modify the slot `B', as it is persistent and
has to be changed inside a transaction.

\begin{Verbatim}[fontsize=\small,frame=leftline,framerule=0.9mm,rulecolor=\color{gray},framesep=5.1mm,xleftmargin=5mm,fontfamily=cmtt]
(setf (protocol-object-b (store-object-with-id 14)) 4)
; => Error
\end{Verbatim}
An object can be removed from the datastore using the transaction
`DELETE-OBJECT', which calls the method `DESTROY-OBJECT' on the
object. Special actions at deletion time have to be added by
overriding `DESTROY-OBJECT'. The basic action is to remove the
object from all its indices.


\subsection{ Snapshotting an object datastore}
We can snapshot the persistent state of all created objects by
using `SNAPSHOT'.

\begin{Verbatim}[fontsize=\small,frame=leftline,framerule=0.9mm,rulecolor=\color{gray},framesep=5.1mm,xleftmargin=5mm,fontfamily=cmtt]
(snapshot)
; Warning:  Backup of the datastore in
; /tmp/object-store/20041112T153046/.
; Warning:
;   Snapshotting subsystem #<STORE-OBJECT-SUBSYSTEM {49396ED5}>
;    of #<MP-STORE DIR: "/tmp/object-store/">...
; Warning:
;   Successfully snapshotted #<STORE-OBJECT-SUBSYSTEM {49396ED5}>
;   of #<MP-STORE DIR: "/tmp/object-store/">.
; => NIL
\end{Verbatim}
This will create a backup directory containing the old transaction
log, and the creation of a snapshot file in the ``current''
directory.

\begin{Verbatim}[fontsize=\small,frame=leftline,framerule=0.9mm,rulecolor=\color{gray},framesep=5.1mm,xleftmargin=5mm,fontfamily=cmtt]
(directory "/tmp/object-store/**/*.*")
; => (#p"/tmp/object-store/20041112T153046/"
;  #p"/tmp/object-store/20041112T153046/transaction-log"
;  #p"/tmp/object-store/current/"
;  #p"/tmp/object-store/current/store-object-subsystem-snapshot")
\end{Verbatim}
The snapshot file contains all persistent objects present at
snapshotting time, and the value of their persistent
slots. Further transaction are recorded in a new transaction log.


\subsection{ Adding indices to store objects}
The object datastore builds upon the functionality of the BKNR
indices system. All store objects are of the metaclass
`INDEXED-CLASS', so adding indices is seamless. Indices are
transient, and are rebuilt every time the datastore is
restored. Adding an index on a transient slot or on a persistent
slot makes no difference.

\begin{Verbatim}[fontsize=\small,frame=leftline,framerule=0.9mm,rulecolor=\color{gray},framesep=5.1mm,xleftmargin=5mm,fontfamily=cmtt]
(define-persistent-class gorilla ()
  ((name :read :index-type string-slot-index
    :index-reader gorilla-with-name
    :index-values all-gorillas)
   (mood :read :index-type keyword-index
    :index-reader gorillas-with-mood
    :index-keys all-gorilla-moods)))

(make-object 'gorilla :name "lucy" :mood :aggressive)
; => #<GORILLA ID: 17>
(make-object 'gorilla :name "john" :mood :playful)
; => #<GORILLA ID: 18>
(make-object 'gorilla :name "peter" :mood :playful)
; => #<GORILLA ID: 19>
(gorilla-with-name "lucy")
; => #<GORILLA ID: 17>
(gorillas-with-mood :playful)
; => (#<GORILLA ID: 19> #<GORILLA ID: 18>)
\end{Verbatim}


\subsection{ Exporting store objects to XML}
Exporting store objects to XML is not possible right now, but it
will soon be available in the BKNR Framework. Stay tuned.


\subsection{ Adding blobs}
A blob is a Binary Large OBject, that means it is a normal
persistent object with an associated binary data (that most of the
time is quite large). The object datastore supports storing this
large binary data outside the transaction log and the snapshot
file in order not to strain the store memory footprint too much,
and to be able to access the binary data from outside the LISP
session. This can be useful in order to copy the binary data using
the operating system calls directly. Blobs are used to store
images in the BKNR Web Framework (in fact, eboy.com contains more
than 40000 images). They have also been used to store MP3 files
for the GPN interactive DJ.

In addition to the binary data, a blob object also holds a `TYPE'
and a `TIMESTAMP'. The type of a blob object is a keyword somehow
identifying the type of binary data it stores. For example, for
the images of the eboy datastore, we used the keywords `:JPEG',
`:PNG', `:GIF' to identity the different file formats used to
store images. The timestamp identifies the time of creation of the
blob object (this can be useful to cache binary data of blob
objects in a web server context).

Stores are implemented in a custom subsystem, which takes as key
argument `:DIRECTORY' the name of a directory where the binary
data of the blob objects is stored as a simple file. This
directory can be further partitioned dynamically by the datastore,
when provided with the argument `:N-BLOBS-PER-DIRECTORY'. The
value of this argument is stored in the directory of the
datastore, so that a future instance of the blob subsystem is
initialised correctly.

We can now add blob support to our existing object datastore by
adding the blob subsystem to its list of subsystems.

\begin{Verbatim}[fontsize=\small,frame=leftline,framerule=0.9mm,rulecolor=\color{gray},framesep=5.1mm,xleftmargin=5mm,fontfamily=cmtt]
(make-instance 'mp-store :directory "/tmp/object-store/"
          :subsystems (list
             (make-instance 'store-object-subsystem)
             (make-instance 'blob-subsystem)))
\end{Verbatim}
The blob subsystem provides a few functions and transactions to
work with blobs. To show how to use these functions, we will
define a blob class in our example store. A photo is simply a
binary object with a name.

\begin{Verbatim}[fontsize=\small,frame=leftline,framerule=0.9mm,rulecolor=\color{gray},framesep=5.1mm,xleftmargin=5mm,fontfamily=cmtt]
(define-persistent-class photo (blob)
  ((name :read)))
\end{Verbatim}
A blob can be created using the function `MAKE-BLOB-FROM-FILE',
which is a wrapper around `TX-MAKE-OBJECT' and the function
`BLOB-FROM-FILE'. The method `BLOB-FROM-FILE' fills the binary
data of a blob object by reading the content of a file. This
binary data is then stored in a file named after the ID of the
object in the blob root directory of the blob subsystem.

\begin{Verbatim}[fontsize=\small,frame=leftline,framerule=0.9mm,rulecolor=\color{gray},framesep=5.1mm,xleftmargin=5mm,fontfamily=cmtt]
(make-blob-from-file "/tmp/bla.png" 'photo :name "foobar"
           :type :png)
; => #<PHOTO ID: 16, TYPE: png>
\end{Verbatim}
We can work with the photo object in the same way as when we work
with a normal object. However, we can access the binary data using
the methods `BLOB-PATHNAME', which returns the pathname to the
file in the blob root that holds the binary data of the
object.

\begin{Verbatim}[fontsize=\small,frame=leftline,framerule=0.9mm,rulecolor=\color{gray},framesep=5.1mm,xleftmargin=5mm,fontfamily=cmtt]
(blob-pathname (store-object-with-id 16))
; => #p"/tmp/object-store/blob-root/16"
\end{Verbatim}
The method `BLOB-TO-FILE' and `BLOB-TO-STREAM' write the binary
data of the object to the specified file or stream (the stream has
to be of the type `(UNSIGNED-BYTE 8)'). The macro `WITH-OPEN-BLOB'
is provided as wrapper around the `WITH-OPEN-FILE' macro.


\subsection{ Relaxed references}
It sometimes happens that a persistent object is deleted while it
still is referenced by another object. This can lead to problems
when snapshotting and restoring the datastore, as the referenced
object is not available anymore.

When a slot is specified as being a relaxed object reference slot
using the slot option `:RELAXED-OBJECT-REFERENCE', a reference to
an unexistent object can be encoded during snapshot. The object
subsystem issues a warning when a reference to a non-existent
object is encoded. When a reference to a deleted object is decoded
form the snapshot file, a `NIL' value is returned if the slot from
where the object is referenced supports relaxed references. Else,
an error is thrown.

\begin{Verbatim}[fontsize=\small,frame=leftline,framerule=0.9mm,rulecolor=\color{gray},framesep=5.1mm,xleftmargin=5mm,fontfamily=cmtt]
(define-persistent-class relaxed-object ()
  ((a :update :relaxed-object-reference t)))

(make-object 'relaxed-object)
; => #<RELAXED-OBJECT ID: 20>
(make-object 'relaxed-object)
; => #<RELAXED-OBJECT ID: 21>
(change-slot-values (store-object-with-id 19)
          'a (store-object-with-id 20))
; => NIL
(delete-object (store-object-with-id 20))
; => T
(snapshot)
; Warning:
;    Encoding reference to destroyed object with ID 20
;    from slot A of object RELAXED-OBJECT with ID 19.
; => NIL
(restore)
; Warning:  restoring #<MP-STORE DIR: "/tmp/object-store/">
; Warning:
;    loading snapshot file
;    /tmp/object-store/current/store-object-subsystem-snapshot
; Warning:
;    Reference to inexistent object with id 20 in
; relaxed slot A of object with class RELAXED-OBJECT with ID 19.
; => :NORMAL
(relaxed-object-a (store-object-with-id 19))
; => NIL
\end{Verbatim}


\section{ Converting from the old datastore}
If you have an existing datastore that has been created with the
first version of the BKNR datastore, you can use a conversion
datastore to convert your data. The conversion datastore is a
specialized version of the new datastore that can load the
snapshot and the transaction log of older datastores, and can
write a new snapshot file.
Furthermore, most index handling code in the applications using
the old datastore can be refactored to use the new `BKNR-INDICES'
facility. For example, the definition of the class `USER' for the
old datastore is:

\begin{Verbatim}[fontsize=\small,frame=leftline,framerule=0.9mm,rulecolor=\color{gray},framesep=5.1mm,xleftmargin=5mm,fontfamily=cmtt]
(define-persistent-class user ()
  ((full-name   :string :update :initform "")
   (last-login  :ulong :update :initform 0)
   (email       :string :update :initform "")
   (login       :string :update :index)
   (password    :string :update :initform "")
   (flags       '(:keyword) :update :keyword :initform nil)
   (preferences :hash :read
      :initform (make-hash-table :test #'eq))
   (subscriptions '(:subscriptions) :update
        :initform nil)
   (mail-error 'mail :update)))

(defun find-user
    (login &key (store (session-store *current-session*)))
  (unless store
    (error "No current datastore session"))
  (index-get store :user-login-index login))
\end{Verbatim}
As you can see, all the index lookup code is written down
explicitely. The class `USER' can be rewritten for the new
datastore:

\begin{Verbatim}[fontsize=\small,frame=leftline,framerule=0.9mm,rulecolor=\color{gray},framesep=5.1mm,xleftmargin=5mm,fontfamily=cmtt]
(define-persistent-class user ()
  ((login :update
     :index-type string-slot-index
     :index-reader find-user :index-values all-users)
   (flags :update :initform nil
     :index-type keyword-list-index
     :index-reader get-flag-users :index-keys all-user-flags)
   
   (email       :update :initform "")
   (full-name   :update :initform "")
   (last-login  :update :initform 0)
   (password    :update :initform "")
   (preferences :read
      :initform (make-hash-table :test #'eq))
   (subscriptions :update :initform nil)
   (mail-error :update)))
\end{Verbatim}
However, the conversion facility has to be used with care. Make
sure that the persistent class definitions of the new datastore
are compatible with the old definitions, and that all the
transactions used in the transaction log are still defined in the
new datastore.
This creates a store that loads its persistent state from old
transaction log and snapshot files. After loading the store, all
that is left to do is to snapshot it, which saves the data in the
new snapshot format. Then, the normal `MP-STORE' class can be used.

\begin{Verbatim}[fontsize=\small,frame=leftline,framerule=0.9mm,rulecolor=\color{gray},framesep=5.1mm,xleftmargin=5mm,fontfamily=cmtt]
(make-instance 'convert-store :directory "/tmp/old-datastore/")
(snapshot)
(close-store)
(make-instance 'mp-store :directory "/tmp/old-datastore/")
\end{Verbatim}


\section{ Store internals}


\subsection{ Binary data files}

This implementation of the BKNR datastore uses a binary encoding
of Lisp data. The encoding library is used by both the transaction
system and the object system and is mostly independent of
them. Users need not be aware of the details of this encoding,
except that (1) primitive data stored needs to be supported by the
encoding library and (2) user-defined object systems need to
register their own encoder and decoder methods to allow their
objects to be used as part of transaction arguments.

\begin{Verbatim}[fontsize=\small,frame=leftline,framerule=0.9mm,rulecolor=\color{gray},framesep=5.1mm,xleftmargin=5mm,fontfamily=cmtt]
Function ENCODE (OBJECT STREAM)

Function DECODE (STREAM) =<  OBJECT
\end{Verbatim}
The `STREAM' must be specialized on `(unsigned-byte 8)'.
The object store subsystem uses the encoding library to encode the
persistent state of all the objects in the store. It does this by
first serializing the layout of a class (which is a list of
slot-names), then by first serializing the class and the id of
each object, and finally by serializing the slots of each
object. This two-step system is necessary to correctly serialize
circular of forward references.

When the snapshot is loaded, an empty instance of each object is
created, and can be referenced only using the `ID'. After each
object has been instantiated, it can be referenced by another
object. The objects are serialized in the order they have been
created.


\subsection{ Datastore session state}
Store sessions are a kind of leftover from the previous version of
the datastore. The old version allowed for multiple sessions to
multiple datastores, which turned out to be very
confusing. However, a session also had a special state. For
example, when a session only needed to read data from the
datastore, and wasn't allowed to modify the persistent state, the
session was in the state `:READ-ONLY'. In the new datastore, the
session is just a special variable called `*CURRENT-SESSION*'. In
the normal case, the variable is set to `:NORMAL'. However, when a
transaction is executed, the `*CURRENT-SESSION*' variable is bound
to `:TRANSACTION'. Application code can thus check if it is
executed inside a transaction (for example, setting a persistent
slot of an object checks if the modification is made inside a
transaction. If it isn't, an error is signalled).
Other special session states are `:RESTORE', which is set when the
store is being restored (transaction functions have to be called,
but not logged to the transaction log). The state
`:ONLY-TRANSIENT' is set when only transient code is allowed to
run. Every attempt to run a transaction signals an error. This is
used when restoring subsystems, which don't have the right to run
transactions.


\subsection{ Transactions}

Transactions are objects of the class `TRANSACTION', and have a
slot containing the symbol of their transaction function, as well
as a list of the arguments that have to be passed to this
function. When a transaction is executed, a timestamp
of the execution time is stored in the object.
In order to execute and log a transaction, the macro
`WITH-STORE-TRANSACTION' has to be used. This macro calls its body
after having executed the transaction passed as arguments. In
fact, the method `EXECUTE' used to execute transactions consists
of a simple use of `WITH-STORE-TRANSACTION':

\begin{Verbatim}[fontsize=\small,frame=leftline,framerule=0.9mm,rulecolor=\color{gray},framesep=5.1mm,xleftmargin=5mm,fontfamily=cmtt]
(defmethod execute ((transaction transaction))
  "Execute TRANSACTION on STORE."
  (with-store-transaction (res transaction)
    res))
\end{Verbatim}
The macro checks if the store is open, if transactions are allowed
in the current session, then acquires the `LOG-GUARD' to serialize
concurrent transactions, executes the transaction function and
writes the transaction to the transaction log file:

\begin{Verbatim}[fontsize=\small,frame=leftline,framerule=0.9mm,rulecolor=\color{gray},framesep=5.1mm,xleftmargin=5mm,fontfamily=cmtt]
(defun invoke-with-store-transaction (fn transaction)
  "Call FN in the context of the transaction TRANSACTION
with the result of transaction execution as the only
argument."
  (cond
    ((eq *current-session* :only-transient)
     (error "Can't execute a transaction when only transient
             operations are allowed."))
    ((in-transaction-p)
     (funcall fn (execute-unlogged transaction)))
    (t
     (unless (eq (store-state *store*) :normal)
       (error (make-condition 'store-not-open)))
     (let* ((*current-session* :transaction)
            (res
             (with-store-guard ()
               (funcall fn (execute-unlogged transaction)))))
       (with-log-guard ()
         (let ((out (store-transaction-log-stream *store*)))
           (encode transaction out)
           (unless *disable-sync*
             (fsync out))))
       res))))
\end{Verbatim}


\subsection{ Snapshot and restore procedures}
When the datastore is snapshotted, the transaction layer ensures
that the store is opened, and that there are subsystems in the
store. Without subsystems, the transaction log is the only way for
the store to achieve persistence, and no snapshot can be made. The
store is then switched to read-only, and a backup directory is
created, containing the current transaction log and previous
snapshot files. This way, the older state of the datastore is not
lost. Then, each subsystem is asked to save its persistent state
by calling the method `SNAPSHOT-SUBSYSTEM'. When an error is
thrown during the snapshot of the subsystems, the backup directory
is renamed to be the current directory, and the store is
reopened.
When the datastore is restored, the store is switched to read
only, and each subsystem is asked to restore its persistent
state. Note that the subsystems are restored in the order in which
they are listed in the `SUBSYSTEMS' slot of the store, so that
dependent subsystems are restored last. When an error is thrown
while restoring the subsystems, the store is closed, and already
opened subsystems are closed using the method
`CLOSE-SUBSYSTEM'. After the restoring of all the subsystems, the
transaction log file is read, and each transaction recorded is
executed. This is where the `UNTIL' parameter comes into
play. Transactions that have been executed after the time of
`UNTIL' are discarded.


\subsection{ Filesystem syncing}
By default, the transaction log file is synced after a transaction
has been executed, so that all the data is correctly written on
disk. However, this can be a major performance stopper when
executing a big batch of transactions (for example, deleting a few
thousands objects). You can disable the mandatory syncing by
executing your transactions inside the form `WITHOUT-SYNC'.

\begin{Verbatim}[fontsize=\small,frame=leftline,framerule=0.9mm,rulecolor=\color{gray},framesep=5.1mm,xleftmargin=5mm,fontfamily=cmtt]
(without-sync ()
   (execute-a-lot-of-transactions))
\end{Verbatim}


\subsection{ Snapshotting and restoring the object subsystem}
Snapshotting and restoring the object subsystem is a bit tricky,
as additional systems come into play. When the object subsystem is
snapshotted using the method `SNAPSHOT-SUBSYSTEM', a snapshot file
containing a binary dump of all the current store objects is
created. First, the layouts of the objects (the name of their
slots) is stored, and minimal information about each object is
stored in the order of their creation (the minimal information
consists of the class of the object, and its ID). After this
information has been stored for each object, the slot values of
the store objects (again in the order of their creation) are
stored in the snapshot file. These two phases are necessary to
allow the snapshotting of circular or forward-referencing
structures.
When the object subsystem is restored, all the indices for classes
contained in the store are cleared in order to accomodate for the
new objects. Be very careful when using class indices that are not
related to store objects. The ID counter of the store subsystem is
reset to 0, and the class-layouts are read from the snapshot
file. Then, the minimal information for each object is read, and
an ``empty version'' of each object is instantiated. Thus, the
objects can be referenced by their ID. Then, the slot values for
each object are read from the snapshot file, references are
resolved (check the section about relaxed references). Finally,
after each slot value has been set, the method
`INITIALIZE-TRANSIENT-INSTANCE' is called for each created
object. The method `INITIALIZE-PERSISTENT-INSTANCE' is not called,
as it has to be executed only once at the time the persistent
object is created.


\subsection{ Garbage collecting blobs}
The binary data of deleted blob objects is kept in the blob root
directory by default. If you want to purge the binary data of
deleted objects, you can use the function
`DELETE-ORPHANED-BLOB-FILES'. However, take note that you won't be
able to restore the persistent state anteriour to the deletion of
the blobs, as their binary data is not stored in the transaction
log and not backed up by the snapshot method of the blob
subsystem.


\subsection{ Schema evolution in the datastore}
The transaction log only stores when a transaction is called, and
with which arguments. However, it doesn't store the definition of
the transaction itself. When the transaction definition is
changed, the transaction log may be restored in a different way,
according to the changes made in the code.

In the same way, class definition changes are not recorded in the
transaction log. When a class definition is changed (for example a
slot initform is changed), the existing instances of the class are
updated accordingly. However, when the snapshot is restored in a
future session, the objects may be different than those created at
the last restore.

The only way to cleanly upgrade transaction definitions and class
definitions is to make a snapshot after the changes have been
made. In a future version of the datastore, we hope to store all
the application sourcecode, so that a restore to a certain point
in time does not depend on the latest version of the code. The
object subsystem warns when a class definition is changed, and
urges the developer to make a snapshot of the database. Please be
careful, this can be a pretty tricky source of bugs.
